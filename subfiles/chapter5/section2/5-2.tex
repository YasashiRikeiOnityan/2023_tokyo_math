\documentclass[../../../doc/main]{subfiles}
\begin{document}
    \setcounter{chapter}{5}
    \setcounter{section}{1}
    \section{問題}\label{問題5}
        \begin{tcolorbox}
            整式$f(x)=(x-1)^2(x-2)$を考える。
            \begin{enumerate}
                \item [\kakkoichi] $g(x)$を実数を係数とする整式とし,$g(x)$を$f(x)$で割った余りを$r(x)$とおく。
                $g(x)^7$を$f(x)$で割った余りと$r(x)^7$を$f(x)$で割った余りが等しいことを示せ。
                \item [\kakkoni] $a,~b$を実数とし,$h(x)=x^2+ax+b$とおく。$h(x)^7$を$f(x)$で割った余りを
                $h_1(x)$とおき,$h_1(x)^7$を$f(x)$で割った余りを$h_2(x)$とおく。$h_2(x)$が$h(x)$に等しくなるような
                $a,~b$の組を求めよ。
            \end{enumerate}
        \end{tcolorbox}
\end{document}