\documentclass[../../../doc/main]{subfiles}
\begin{document}
    \setcounter{chapter}{4}
    \setcounter{section}{1}
    \section{問題}\label{問題4}
        \begin{tcolorbox}
            座標空間内の$4$点$\mathrm O(0,\,0,\,0),\mathrm A(2,\,0,\,0),\mathrm B(1,\,1,\,1),\mathrm C(1,\,2,\,3)$を考える。
            \begin{enumerate}
                \item [\kakkoichi]
                $\bekutoru{OP}\perp\bekutoru{OA},\bekutoru{OP}\perp\bekutoru{OB},\bekutoru{OP}\cdot\bekutoru{OC}=1$を満たす点$\mathrm P$の座標を求めよ。
                \item [\kakkoni]
                点$\mathrm P$から直線$\mathrm{AB}$に垂線を下ろし,その垂線と直線$\mathrm{AB}$の交点を$\mathrm H$とする。$\bekutoru{OH}$を$\bekutoru{OA}$と$\bekutoru{OB}$を用いて表せ。
                \item [\kakkosan]
                点$\mathrm Q$を$\bekutoru{OQ}=\bunsuu34\bekutoru{OA}+\bekutoru{OP}$により定め,$\mathrm Q$を中心とする半径$r$の球面$S$を考える。$S$が三角形$\mathrm{OHB}$と共有点を持つような$r$の範囲を求めよ。ただし,三角形$\mathrm{OHB}$は$3$点$\mathrm O,\,\mathrm H,\,\mathrm B$を含む平面内にあり,周とその内部からなるものとする。
            \end{enumerate}
        \end{tcolorbox}
\end{document}