\documentclass[../../../doc/main]{subfiles}
\begin{document}
    \setcounter{chapter}{6}
    \setcounter{section}{1}
    \section{問題}\label{問題6}
        \begin{tcolorbox}
            Oを原点とする座標空間において,不等式$\emabs{x}\leqq1,~\emabs{y}\leqq1,~\emabs{z}\leqq1$の表す立方体を考える。その立方体の表面のうち,$z<1$を満たす部分を$S$とする。\par
            以下,座標空間内の2点A,Bが一致するとき,線分ABは点Aを表すものとし,その長さを0と定める。
            \begin{enumerate}
                \item [\kakkoichi] 
                    座標空間内の点Pが次の条件(i),(ii)をともに満たすとき,点Pが動きうる範囲$V$の体積を求めよ。
                \begin{enumerate}
                    \item [\tokeiichi] 
                        $\mathrm{OP}\leqq\sqrt{3}$
                    \item [\tokeini] 
                        線分OPと$S$は,共有点を持たないか,点Pのみを共有点に持つ。
                \end{enumerate}
                \item [\kakkoni] 
                    座標空間内の点Nと点Pが次の条件(iii),(iv),(v)をすべて満たすとき,点Pが動きうる範囲$W$の体積を求めよ。必要ならば,$\sin{\alpha}=\bunsuu{1}{\sqrt{3}}$を満たす実数$\alpha~~\left(0<\alpha<\bunsuu{\pi}{2}\right)$を用いてよい。
                    \begin{enumerate}
                        \item [\tokeisan]
                            $\mathrm{ON}+\mathrm{NP}\leqq\sqrt{3}$
                        \item [\tokeishi]
                            線分ONと$S$は共有点を持たない。
                        \item [\tokeigo]
                            線分NPと$S$は,共有点をもたないか,点Pのみを共有点に持つ。
                    \end{enumerate}
            \end{enumerate}
        \end{tcolorbox}
\end{document}