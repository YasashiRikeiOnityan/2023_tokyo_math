\documentclass[../../../doc/main]{subfiles}
\begin{document}
    \setcounter{chapter}{2}
    \setcounter{section}{2}
    \section{解説}\label{解説2}
    \begin{enumerate}
        \item [\kakkoichi]
            \textcolor{myBlue2}{確率は$\text{(条件を満たす場合の数)}/\text{(全事象の数)}$で求まりますので,それらを求めていきましょう.} \\
            全事象である$12$個の玉の並べ方は$\kumiawase{12}{3}\sdot\kumiawase{9}{4}$通りです。 \\
            \textcolor{myBlue2}{もしくは$\bunsuu{12!}{3!\sdot4!\sdot5!}$ですね.また,$12$個の玉の並べ方のうちのそれぞれが現れる確率は,どの並べ方でも等しく$\bunsuu{1}{\kumiawase{12}{3}\sdot\kumiawase{9}{4}}$となります.
            同様に確からしいというやつですね.この記述はあると丁寧でしょう.}\\
            どの赤玉も隣り合わない並べ方は,黒玉と白玉合計$8$個をまず並べ,その両端または隙間の$9$箇所に重複がないように赤玉$4$個を入れることで得られます.
            \begin{figure}[htbp]
                \centering
                \begin{tikzpicture}
                    \foreach \t in {-3.5,-2.5,1.5} {
                        \fill [black] (\t,0) circle(.3cm);
                        \draw [black] (\t,0) circle(.3cm);
                    }
                    \foreach \t in {-1.5,-0.5,0.5,2.5,3.5} {
                        \fill [white] (\t,0) circle(.3cm);
                        \draw [black] (\t,0) circle(.3cm);
                    }
                    \foreach \t in {-4,-3,...,4} {
                        \draw (\t,-.6) node {\,\rotatebox{90}{$\to$}};
                    }
                    \foreach \t in {-2,0,1,4} {
                        \fill [red] (\t,-1.2) circle(.3cm);
                        \draw [black] (\t,-1.2) circle(.3cm);
                    }
                    \draw (4.35,0) node [right] {$\leftarrow~\kumiawase{8}{3}$通り};
                    \draw (4.35,-1.2) node [right] {$\leftarrow~\kumiawase{9}{4}$通り};
                \end{tikzpicture}
            \end{figure}\\
            よって,求める確率$p$は
            \begin{align*}
            p=&\bunsuu{\kumiawase{8}{3}\sdot\kumiawase{9}{4}}{\kumiawase{12}{3}\sdot\kumiawase{9}{4}}=\bunsuu{8\sdot7\sdot6}{12\sdot11\sdot10}=\boldsymbol{\bunsuu{14}{55}}\kotae
            \end{align*}
        \item [\kakkoni]
            赤玉が隣り合わず,さらに黒玉も隣り合わない並べ方の総数$ N_3 $を考える。(1)で黒玉と白玉だけを並べた段階において,隣り合う黒玉の状況によって場合分けする。
            \begin{enumerate}
                \item [\tokeiichi]
                    どの黒玉も隣り合わない場合
                    黒玉と白玉の並べ方は,白玉$ 5 $個の間と両端の合計$ 6 $箇所に重複がないように黒玉$ 3 $個を入れればよいので$ \kumiawase{6}{3} = 20 $通りである。
                    この間のどこに赤玉を入れても黒玉が隣接することはないので,赤玉まで含めた並べ方は$ 20 \sdot \kumiawase{9}{4} = 20 \sdot 126 = 2520 $通りである。
                \item [\tokeini]
                    $ 2 $個の黒玉が隣り合い,残りの$ 1 $個とは隣り合っていない場合
                    白玉$ 5 $個の間と両端の合計$ 6 $箇所のうち相異なる箇所に,「$ 2 $個の黒玉」と「$ 1 $個の黒玉」を1組ずつ入れることで黒玉と白玉の並べ方が作れる。この並べ方は$ 6 \sdot 5 = 30 $通りである。
                    「2個の黒玉」の間には赤玉を入れる必要があり,それ以外の$ 8 $箇所には残り$ 3 $個の赤玉をどこに入れても黒玉は隣接しないので,赤玉まで含めた並べ方は$ 30 \sdot \kumiawase{8}{3} = 30 \sdot 56 = 1680 $通りである。
                \item [\tokeisan]
                    $ 3 $個の黒玉が連続して並ぶ場合
                    白玉$ 5 $個の間と両端の合計$ 6 $箇所のうち$ 1 $箇所に,「3個の黒玉」を入れることで黒玉と白玉の並べ方が作れる。この並べ方は$ 6 $通りである。
                    「3個の黒玉」の間$ 2 $箇所にはどちらも赤玉を入れる必要があり,それ以外の$ 7 $箇所には残り$ 2 $個の赤玉をどこに入れても黒玉は隣接しないので,赤玉まで含めた並べ方は$ 6 \sdot \kumiawase{7}{2} = 6 \sdot 21 = 126 $通りである。
            \end{enumerate}
        よって,$ N_3 = 2520 + 1680 + 126 = 4326 $であるから,
        \begin{align*}
            q = \bunsuu{N_3}{N_2} = \bunsuu{4326}{7056} = \boldsymbol{\bunsuu{103}{168}} \kotae
        \end{align*}
    \end{enumerate}
\end{document}