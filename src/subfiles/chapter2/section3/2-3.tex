\documentclass[../../../doc/main]{subfiles}
\begin{document}
    \setcounter{chapter}{2}
    \setcounter{section}{2}
    \section{解説}\label{解説2}
    \begin{enumerate}
        \item [\kakkoichi]
            $ 12 $個の玉の並べ方は$\kumiawase{12}{3} \sdot \kumiawase{9}{4} = \bunsuu{12!}{3!9!} \sdot \bunsuu{9!}{4!5!}$通りである。この値を$ N_1 $とする。また,$ 12 $個の玉の並べ方のうちのそれぞれが現れる確率はどの並べ方でも等しく$\bunsuu{1}{N_1}$である。\\
            どの赤玉も隣り合わない並べ方は,黒玉と白玉合計$ 8 $個をまず並べ,その両端または隙間の$ 9 $箇所に重複がないように赤玉$ 4 $個を入れることで得られる。したがって,その総数を$ N_2 $とすると,
            \begin{align*}
                N_2 = \kumiawase{8}{3} \sdot \kumiawase{9}{4} = \bunsuu{8!}{3!5!} \sdot \bunsuu{9!}{4!5!}
            \end{align*}
            である。よって求める確率$ p $は
            \begin{align*}
            p =& \bunsuu{N_2}{N_1} = \bunsuu{8!}{3!5!} \sdot \bunsuu{9!}{4!5!} \sdot \bunsuu{3!9!}{12!} \sdot \bunsuu{4!5!}{9!} = \bunsuu{8!9!}{5!12!} = \bunsuu{6 \sdot 7 \sdot 8 \sdot 9}{9 \sdot 10 \sdot 11 \sdot 12} = \boldsymbol{\bunsuu{14}{55}} \kotae
            \end{align*}
        \item [\kakkoni]
        赤玉が隣り合わず,さらに黒玉も隣り合わない並べ方の総数$ N_3 $を考える。(1)で黒玉と白玉だけを並べた段階において,隣り合う黒玉の状況によって場合分けする。
            \begin{enumerate}
            \item [\tokeiichi]
                どの黒玉も隣り合わない場合
                黒玉と白玉の並べ方は,白玉$ 5 $個の間と両端の合計$ 6 $箇所に重複がないように黒玉$ 3 $個を入れればよいので$ \kumiawase{6}{3} = 20 $通りである。
                この間のどこに赤玉を入れても黒玉が隣接することはないので,赤玉まで含めた並べ方は$ 20 \sdot \kumiawase{9}{4} = 20 \sdot 126 = 2520 $通りである。
            \item [\tokeini]
                $ 2 $個の黒玉が隣り合い,残りの$ 1 $個とは隣り合っていない場合
                白玉$ 5 $個の間と両端の合計$ 6 $箇所のうち相異なる箇所に,「$ 2 $個の黒玉」と「$ 1 $個の黒玉」を1組ずつ入れることで黒玉と白玉の並べ方が作れる。この並べ方は$ 6 \sdot 5 = 30 $通りである。
                「2個の黒玉」の間には赤玉を入れる必要があり,それ以外の$ 8 $箇所には残り$ 3 $個の赤玉をどこに入れても黒玉は隣接しないので,赤玉まで含めた並べ方は$ 30 \sdot \kumiawase{8}{3} = 30 \sdot 56 = 1680 $通りである。
            \item [\tokeisan]
                $ 3 $個の黒玉が連続して並ぶ場合
                白玉$ 5 $個の間と両端の合計$ 6 $箇所のうち$ 1 $箇所に,「3個の黒玉」を入れることで黒玉と白玉の並べ方が作れる。この並べ方は$ 6 $通りである。
                「3個の黒玉」の間$ 2 $箇所にはどちらも赤玉を入れる必要があり,それ以外の$ 7 $箇所には残り$ 2 $個の赤玉をどこに入れても黒玉は隣接しないので,赤玉まで含めた並べ方は$ 6 \sdot \kumiawase{7}{2} = 6 \sdot 21 = 126 $通りである。
            \end{enumerate}
        よって,$ N_3 = 2520 + 1680 + 126 = 4326 $であるから,
        \begin{align*}
            q = \bunsuu{N_3}{N_2} = \bunsuu{4326}{7056} = \boldsymbol{\bunsuu{103}{168}} \kotae
        \end{align*}
    \end{enumerate}
\end{document}