\documentclass[../../../doc/main]{subfiles}
\begin{document}
    \setcounter{chapter}{3}
    \setcounter{section}{1}
    \section{問題}\label{問題3}
        \begin{tcolorbox}
            $a$を実数とし,座標平面上の点$(0,\,a)$を中心とする半径$1$の円の周を$C$とする。
            \begin{enumerate}
                \item [\kakkoichi] 
                    $C$が,不等式$y>x^2$の表す領域に含まれるような$a$の範囲を求めよ。
                \item [\kakkoni] 
                    $a$は$\kakkoichi$で求めた範囲にあるとする。$C$のうち$x\geqq0$かつ$y<a$を満たす部分を$S$とする。$S$上の点Pに対し,点Pでの$C$の接線が放物線$y=x^2$によって切り取られてできる線分の長さを$L_{\text{P}}$とする。$L_{\text{Q}}=L_{\text{R}}$となる$S$上の相異なる$2$点Q,Rが存在するような$a$の範囲を求めよ。
            \end{enumerate}
        \end{tcolorbox}
\end{document}