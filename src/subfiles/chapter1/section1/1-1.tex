\documentclass[../../../doc/main]{subfiles}
\begin{document}
    \setcounter{chapter}{1}
    \setcounter{section}{0}
    \section{概要}\label{概要1}
        本問の構成は,\kakkoichi で定積分の不等式評価を行い,得られた不等式を元に\kakkoni で極限値を求める流れとなっています.
        このような形式の問題では,方針が立てにくいという点で\kakkoichi の不等式評価の方が難しい傾向にあります.なぜなら大抵の場合,
        不等式評価をした結果に対してはさみうちの原理を適用すれば\kakkoni の極限値は求められるからです.そのため,\kakkoichi 
        をクリアした受験生は\kakkoni もクリアしていくと予想され,点数の差がつく問題であると考えられます.
\end{document}