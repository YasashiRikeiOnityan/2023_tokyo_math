\documentclass[../../../doc/main]{subfiles}
\begin{document}
    \setcounter{chapter}{1}
    \setcounter{section}{3}
    \section{研究}\label{研究1}
        \ref{解説1}節では,式変形により解き進めましたが一体何を計算していたのでしょうか?まず,$A_k$の定積分が
        どの図形の面積を表すかを考えてみましょう.以下の図に$y=\emabs{\sin{(x^2)}}$のグラフを描画します.また,$A_2=\dint{\sqrt{2\pi}}{\sqrt{3\pi}}\emabs{\sin{(x^2)}}\,dx$
        に対応する部分を\textcolor{myBlue}{水色}で塗りつぶします.
        \begin{figure}[htbp]
            \centering
            \begin{tikzpicture}
                \begin{axis}[
                    axis lines=middle,
                    xlabel={$x$},
                    ylabel={$y$},
                    ylabel style={at={(ticklabel* cs:1.00)}, anchor=west},
                    xmax=5.6,
                    xmin=-0.3,
                    ymax=1.2,
                    ymin=-0.3,
                    xtick={0,{sqrt(pi)},{sqrt(2*pi)},{sqrt(3*pi)},{sqrt(4*pi)},{sqrt(5*pi)},{sqrt(6*pi)},{sqrt(7*pi)}},
                    ytick={0,1},
                    xticklabels={0,$\sqrt{\pi}$,$\sqrt{2\pi}$,$\sqrt{3\pi}$,$\sqrt{4\pi}$,$\sqrt{5\pi}$,$\sqrt{6\pi}$,$\sqrt{7\pi}$},
                    yticklabels={0,1},
                    enlargelimits=true,
                    xscale=1.8
                    %grid style=dashed,
                    %grid=both
                ]
                \node [draw=white,anchor=north east] at (axis cs:-0.02,-0.02) {O};
                \addplot [domain=0:4.8,samples=5000] {abs(sin(deg(x^2)))} node[above right] {$y=|\sin{(x^2)}|$};
                %\addplot[white, fill=white] coordinates {(0,0) (5,0) (5,-0.2) (0,-0.2) -- cycle};
                \addplot [domain={sqrt(2*pi)}:{sqrt(3*pi)}, myBlue, thick, fill=myBlue!50,samples=5000] {abs(sin(deg(x^2)))} \closedcycle;
                \end{axis}
            \end{tikzpicture}
        \end{figure}
        $y=\emabs{\sin{(x^2)}}$は$x=\sqrt{m\pi}~(m\text{は0以上の整数})$のときに$y=0$となります.したがって,お山が連なった形状となります.そのお山を左から順に$0\text{番目},~1\text{番目},~\udots$としていくと,
        $k$番目のお山の面積が$A_k$となるわけです.
\end{document}