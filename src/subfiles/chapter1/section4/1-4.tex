\documentclass[../../../doc/main]{subfiles}
\begin{document}
    \setcounter{chapter}{1}
    \setcounter{section}{3}
    \section{研究}\label{研究1}
        \subsection{$A_k$は何を表しているか?}\label{Akは何を表しているか}
        \ref{解説1}節では,式変形により解き進めましたが,一体何を計算していたのでしょうか?まず,$A_k$の定積分が
        どの図形の面積を表すかを考えてみましょう.以下の図\ref{1.4.1_1}に$y=\emabs{\sin{(x^2)}}$のグラフを描画します.
        \begin{figure}[htbp]
            \centering
            \begin{tikzpicture}
                \begin{axis}[
                    axis lines=middle,
                    xlabel={$x$},
                    ylabel={$y$},
                    ylabel style={at={(ticklabel* cs:1.00)}, anchor=west},
                    xmax=5.6,
                    xmin=-0.3,
                    ymax=1.2,
                    ymin=-0.3,
                    xtick={0,{sqrt(pi)},{sqrt(2*pi)},{sqrt(3*pi)},{sqrt(4*pi)},{sqrt(5*pi)},{sqrt(6*pi)},{sqrt(7*pi)}},
                    ytick={0,1},
                    xticklabels={0,$\sqrt{\pi}$,$\sqrt{2\pi}$,$\sqrt{3\pi}$,$\sqrt{4\pi}$,$\sqrt{5\pi}$,$\sqrt{6\pi}$,$\sqrt{7\pi}$},
                    yticklabels={0,1},
                    enlargelimits=true,
                    xscale=1.8
                    %grid style=dashed,
                    %grid=both
                ]
                \node [draw=white,anchor=north east] at (axis cs:-0.02,-0.02) {O};
                \addplot [domain=0:4.8,samples=5000] {abs(sin(deg(x^2)))} node[above right] {$y=|\sin{(x^2)}|$};
                %\addplot[white, fill=white] coordinates {(0,0) (5,0) (5,-0.2) (0,-0.2) -- cycle};
                \addplot [domain={sqrt(2*pi)}:{sqrt(3*pi)}, myBlue, thick, fill=myBlue!50,samples=5000] {abs(sin(deg(x^2)))} \closedcycle;
                \end{axis}
            \end{tikzpicture}
            \caption{$y=\emabs{\sin{(x^2)}}$の概形}
            \label{1.4.1_1}
        \end{figure}
        $y=\emabs{\sin{(x^2)}}$は$x=\sqrt{m\pi}~(m\text{は0以上の整数})$のときに$y=0$となり,お山が連なった形状となります.そのお山を左から順に$0\text{番目},~1\text{番目},~\udots$としていくと,
        $k$番目のお山の面積が$A_k$となるわけです.例として,\textcolor{myBlue}{水色}で塗りつぶした部分が$A_2=\dint{\sqrt{2\pi}}{\sqrt{3\pi}}\emabs{\sin{(x^2)}}\,dx$
        に対応しています.
        \subsection{$A_k$を図形的に評価する}\label{Akを図形的に評価する}
        \ref{Akは何を表しているか}項で$A_k$が何を表しているかが分かりました.では,
        以下の図のように下から評価する\footnote{「下から評価する」とは,$x\leq A_k$もしくは$x<A_k$を満たす$x$を求めることです.同様に「上から評価する」という言い回しもあります.}ことはできないでしょうか?$k=2$の場合を考えてみましょう.

        \begin{figure}[htbp]
            \centering
            \begin{tikzpicture}
                \begin{axis}[
                    axis lines=middle,
                    xlabel={$x$},
                    ylabel={$y$},
                    ylabel style={at={(ticklabel* cs:1.00)}, anchor=west},
                    xmax=5.6,
                    xmin=-0.3,
                    ymax=1.2,
                    ymin=-0.3,
                    xtick={0,{sqrt(pi)},{sqrt(2*pi)},{sqrt(3*pi)},{sqrt(4*pi)},{sqrt(5*pi)},{sqrt(6*pi)},{sqrt(7*pi)}},
                    ytick={0,1},
                    xticklabels={0,$\sqrt{\pi}$,$\sqrt{2\pi}$,$\sqrt{3\pi}$,$\sqrt{4\pi}$,$\sqrt{5\pi}$,$\sqrt{6\pi}$,$\sqrt{7\pi}$},
                    yticklabels={0,1},
                    enlargelimits=true,
                    xscale=1.8
                    %grid style=dashed,
                    %grid=both
                ]
                \node [draw=white,anchor=north east] at (axis cs:-0.02,-0.02) {O};
                \addplot [domain=0:4.8,samples=5000] {abs(sin(deg(x^2)))} node[above right] {$y=|\sin{(x^2)}|$};
                \addplot [domain={sqrt(2*pi)}:{sqrt(3*pi)}, myBlue, thick, fill=myBlue!50,samples=5000] {abs(sin(deg(x^2)))} \closedcycle;
                \addplot [black, semithick] coordinates {({sqrt(2*pi)},0) ({sqrt(5)*sqrt(pi/2)},1)};
                \addplot [black, semithick] coordinates {({sqrt(2*pi)},0) ({sqrt(3*pi)},0)};
                \addplot [black, semithick] coordinates {({sqrt(3*pi)},0) ({sqrt(5)*sqrt(pi/2)},1)};
                \end{axis}
            \end{tikzpicture}
            \caption{$A_2$の面積を三角形の面積で下から評価する}
            \label{A2の面積を三角形の面積で下から評価する}
        \end{figure}

        \noindent
        もし
        \begin{align*}
            \text{(三角形の面積)}\leq A_2\naraba \bunsuu{1}{\sqrt{(2+1)\pi}}\leq A_2\quad\quad\text{すなわち}\quad\quad\bunsuu{\sqrt{3\pi}-\sqrt{2\pi}}{2}\leq A_2\naraba \bunsuu{1}{\sqrt{3\pi}}\leq A_2
        \end{align*}
        が真であれば,下から評価できるということになります.要は,議論の流れとして
        \begin{enumerate}
            \item [\tokeiichi] $\bunsuu{\sqrt{3\pi}-\sqrt{2\pi}}{2}\leq A_2$である.
            \item [\tokeini] また,$\bunsuu{1}{\sqrt{3\pi}}\leq\bunsuu{\sqrt{3\pi}-\sqrt{2\pi}}{2}$である.
            \item [\tokeisan] したがって,$\bunsuu{1}{\sqrt{3\pi}}\leq A_2$である.
        \end{enumerate}
        と言いたいわけです.\tokeiichi は図形より正しいとしてよいでしょう.しかしながら,\tokeini に関して
        \begin{align*}
            \bunsuu{\sqrt{3\pi}-\sqrt{2\pi}}{2}-\bunsuu{1}{\sqrt{3\pi}}=\bunsuu{3\pi-\sqrt{6}\pi-2}{2\sqrt{3\pi}}=\bunsuu{(3-\sqrt{6})\pi-2}{2\sqrt{3\pi}}<\bunsuu{(3-2.4)\pi-2}{2\sqrt{3\pi}}<\bunsuu{0.6\sdot 3.2-2}{2\sqrt{3\pi}}<0
        \end{align*}
        であるから,\tokeini は成り立ちません.この例では$k=2$のときを計算しましたが,すべての正の整数$k$について
        同様の結果になります.よって,図形的に下から評価することは難しいでしょう.仮に,何らかの方法で下から評価することができても,
        上から評価するのはさらに難しいでしょう.
        \subsection{もう一度置換!}\label{もう一度置換}
        \ref{解説1}節における\kakkoichi の解説の途中で以下の不等式が得られました.
        \begin{align*}
            \dint{k\pi}{(k+1)\pi}\emabs{\sin{t}}\sdot\bunsuu{dt}{2\sqrt{(k+1)\pi}}\leq A_k \leq \dint{k\pi}{(k+1)\pi}\emabs{\sin{t}}\sdot\bunsuu{dt}{2\sqrt{k\pi}}~\sdots\sdots~\maruichi
        \end{align*}
        そして,$\dint{k\pi}{(k+1)\pi}\emabs{\sin{t}}\,dt=2~\sdots\sdots~\asta$であることを用いて与不等式を証明しました.しかし,この$\asta$は思い付きましたでしょうか?
        思いつかなかった方は続けてお読みください.複雑な積分では,積分区間を簡単にすることが大事な方針の1つとなります.不等式$\maruichi$の両側とも積分区間は$k\pi$から$(k+1)\pi$となっております.
        これを$0$から$\pi$となるように置換してみましょう.
        \begin{align*}
            \begin{tabular}{|c|c|} \hline
                $t$ & $k\pi\to(k+1)\pi\bsityuu$ \\ \hline
                $s$ & $0\to\pi\bsityuu$ \\ \hline
            \end{tabular}
        \end{align*}
        となって欲しいので...~$s=t-k\pi$と置換すればよさそうですね.したがって,$ds=dt$より,不等式$\maruichi$は
        \begin{align*}
            \dint{0}{\pi}\emabs{\sin{(s+k\pi)}}\sdot\bunsuu{ds}{2\sqrt{(k+1)\pi}}\leq A_k \leq \dint{0}{\pi}\emabs{\sin{(s+k\pi)}}\sdot\bunsuu{ds}{2\sqrt{k\pi}}
        \end{align*}
        ここで
        \begin{align*}
            &\emabs{\sin{(s+k\pi)}}=\emabs{\pm\sin{s}}=\emabs{\pm1}\sdot\emabs{\sin{s}}=\emabs{\sin{s}}=\sin{s} \\
            &\chub ~~ k\text{が偶数のとき}\emabs{\sin{(s+k\pi)}}=\sin{s},~k\text{が奇数のとき}\emabs{\sin{(s+k\pi)}}=-\sin{s} \\
            &\chub ~~ \text{積分区間より}0\leq s\leq \pi\text{で},~\sin{s}\geq0\text{なので}\emabs{\sin{s}}=\sin{s}
        \end{align*}
        であるから
        \begin{align*}
            &\bunsuu{1}{2\sqrt{(k+1)\pi}}\dint{0}{\pi}\sin{s}\,ds\leq A_k \leq \bunsuu{1}{2\sqrt{k\pi}}\dint{0}{\pi}\sin{s}\,ds \\
            &\Y \bunsuu{1}{\sqrt{(k+1)\pi}}\leq A_k\leq \bunsuu{1}{\sqrt{k\pi}}~~\left(\nazenara~\dint{0}{\pi}\sin{s}\,ds=\teisekibun{-\cos{s}}{0}{\pi}=-\cos{\pi}-(-\cos{0})=2\right)
        \end{align*}
        が得られます.このように積分区間が簡単になるように置換をしていきましょう.
\end{document}