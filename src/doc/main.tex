\documentclass[10pt,a4paper,dvipdfmx,fleqn]{report}
%%%%%%%%%%%%%%% packages %%%%%%%%%%%%%%%
\usepackage[top=25mm,bottom=25mm,left=20mm,right=20mm]{geometry}
\usepackage{graphicx} 
\usepackage{xcolor}
\usepackage{titlesec} 
\usepackage{lipsum} %just to generate text for the example
\usepackage{amsmath,amsthm,amssymb,esvect,amsfonts,fancybox,tikz,tikz-3dplot,ulem,fancyhdr,proof}
\usepackage{../sty/eclbkbox,okumacro,../sty/ascolorbox,pgfplots,subfiles}
\usetikzlibrary{patterns,intersections,calc,quotes,angles,arrows.meta,positioning,graphs,through,shapes.geometric}
\tcbuselibrary{raster,skins,theorems,breakable}
\allowdisplaybreaks
\usepackage[dvipdfmx]{hyperref}
\usepackage{pxjahyper}
\hypersetup{% hyperrefオプションリスト
    setpagesize=false,
    bookmarksnumbered=true,%
    bookmarksopen=true,%
    colorlinks=true,%
    linkcolor=black,
    citecolor=red,
}
\usepackage{ceo}
\usepackage{epic,eepic,emath,emathW,emathB,emathBk,emathMw,emathCap,emathE}
%%%%%%%%%%%%%%% settings %%%%%%%%%%%%%%%
% =,→ 間の余白
\thickmuskip=1.0\thickmuskip
% +,- 間の余白
\medmuskip=0.8\medmuskip
% … などの装飾記号の余白
\thinmuskip=0.8\thinmuskip
% 行列を詰める
\arraycolsep=0.3\arraycolsep
% 数式の上下のスペースの変更
\AtBeginDocument{
    \abovedisplayskip     =0.5\abovedisplayskip
    \abovedisplayshortskip=0.5\abovedisplayshortskip
    \belowdisplayskip     =0.5\belowdisplayskip
    \belowdisplayshortskip=0.5\belowdisplayshortskip}
\newcommand*\HUGE{\Huge}
\newcommand*\chapnamefont{\normalfont\LARGE\MakeUppercase}
\newcommand*\chapnumfont{\normalfont\HUGE}
\newcommand*\chaptitlefont{\normalfont\HUGE\bfseries}

\newlength\beforechapskip
\newlength\midchapskip
\setlength\midchapskip{\paperwidth}
\addtolength\midchapskip{-\textwidth}
\addtolength\midchapskip{-\oddsidemargin}
\addtolength\midchapskip{-1in}
\setlength\beforechapskip{18mm}

\titleformat{\chapter}[display]
    {\normalfont\filleft}
    {{\chapnamefont\chaptertitlename}%
    \color[rgb]{.53,.8,.92}
    \makebox[20mm][l]{\hspace{.8em}%
    \resizebox{!}{\beforechapskip}{\chapnumfont\thechapter}%
    \hspace{.8em}%
    \rule{\midchapskip}{\beforechapskip}%
    }%
    }%
    {25pt}
    {\chaptitlefont}
\titlespacing*{\chapter}{0pt}{40pt}{40pt}

\definecolor{myBlue}{rgb}{.53,.8,.92}
\definecolor{myBlue2}{rgb}{0,0,.52}
\newcommand{\sectionformat}[1]{%
    \begin{tikzpicture}[baseline=(title.base)]
        \node (title) {\thesection\hspace{1em}#1};
        %\draw[double] (title.west) |- (title.south) (title.north) -| (title.east);
        %\draw[double] (title.south) -| (title.east);
        \fill [myBlue] (title.west) -- ++(0,-.25) -- ++(-.2,0) -- ++(0,.6) -- ++(.2,0) -- cycle;
        \coordinate (A) at ($(title.west)+(0cm,-.25cm)$);
        \coordinate (B) at ($(A)+(10cm,0cm)$);
        \coordinate (C) at ($(A)!(title.east)!(B)$);
        \draw [myBlue,very thick] (A) -- ++(-.2,0) -- (C);
    \end{tikzpicture}
}

\titleformat{\section}{\normalfont\Large\bfseries}{}{0em}{\sectionformat}%

\definecolor{titlepagecolor}{rgb}{.53,.8,.92}
\definecolor{namecolor}{rgb}{.27,.51,.7} 

\newtcbtheorem[number within=section]{mytheo}{定理}%
    {fonttitle=\gtfamily\sffamily\bfseries\upshape,
    colframe=myBlue,colback=myBlue!2!white,
    rightrule=0pt,leftrule=0pt,bottomrule=2pt,
    colbacktitle=myBlue,theorem style=standard,breakable,arc=0pt}{tha}
%%%%%%%%%%%%%%% document %%%%%%%%%%%%%%%
\begin{document}

    \subfile{../subfiles/titlepage/title}

    \tableofcontents
    \setcounter{chapter}{-1}

    \chapter{はじめに}
        \subfile{../subfiles/chapter0/section1/0-1}
        \subfile{../subfiles/chapter0/section2/0-2}

    \chapter{第1問}
        \subfile{../subfiles/chapter1/section1/1-1}
        \subfile{../subfiles/chapter1/section2/1-2}
        \subfile{../subfiles/chapter1/section3/1-3}
        \subfile{../subfiles/chapter1/section4/1-4}
    
    \chapter{第2問}
        \subfile{../subfiles/chapter2/section1/2-1}
        \subfile{../subfiles/chapter2/section2/2-2}
        \subfile{../subfiles/chapter2/section3/2-3}
        \subfile{../subfiles/chapter2/section4/2-4}

    \chapter{第3問}
        \subfile{../subfiles/chapter3/section1/3-1}
        \subfile{../subfiles/chapter3/section2/3-2}
        \subfile{../subfiles/chapter3/section3/3-3}
        \subfile{../subfiles/chapter3/section4/3-4}

    \chapter{第4問}
        \subfile{../subfiles/chapter4/section1/4-1}
        \subfile{../subfiles/chapter4/section2/4-2}
        \subfile{../subfiles/chapter4/section3/4-3}
        \subfile{../subfiles/chapter4/section4/4-4}
    
    \chapter{第5問}
        \subfile{../subfiles/chapter5/section1/5-1}
        \subfile{../subfiles/chapter5/section2/5-2}
        \subfile{../subfiles/chapter5/section3/5-3}
        \subfile{../subfiles/chapter5/section4/5-4}

    \chapter{第6問}
        \subfile{../subfiles/chapter6/section1/6-1}
        \subfile{../subfiles/chapter6/section2/6-2}
        \subfile{../subfiles/chapter6/section3/6-3}
        \subfile{../subfiles/chapter6/section4/6-4}

\end{document}